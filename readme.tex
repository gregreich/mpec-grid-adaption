% TODO: 
% page 19 rewrite SV explanation

\documentclass[12pt]{article}

\usepackage{amsfonts}
\usepackage{amsthm}
\usepackage{chicago}
\usepackage[T1]{fontenc}
%\usepackage{bibgerm}
\usepackage{tabu}
\usepackage{longtable}
\usepackage{latexsym,amssymb,graphicx,array,amsmath}
\usepackage{booktabs}
\usepackage{endnotes}
\usepackage{graphicx}
%\usepackage{natbib}
\usepackage{epic}
\usepackage{setspace}
\usepackage[flushmargin]{footmisc}
\usepackage{graphpap}
\usepackage{rotating}
\usepackage{amssymb}
%\usepackage{wasysym}
\usepackage{multirow}
\usepackage{booktabs}
\usepackage{acronym}
\usepackage{cite}
\usepackage{epstopdf}

%\renewcommand{\bflabel}[1]{\normalfont{\normalsize{#1}}\hfill}
\usepackage[hang]{caption}
\usepackage{float} 

\usepackage{tabularx, booktabs}
%\usepackage{hyperref} 
\def\tablenotes{\vskip2pt\footnotesize}
\let\endtablenotes\relax

\setlength{\textwidth}{6.5in}
\setlength{\textheight}{9.3in}
\setlength{\oddsidemargin}{0in}
\setlength{\evensidemargin}{0in}
\setlength{\topmargin}{-0.75in}
\newtheorem{definition}{Definition}
\newtheorem{theorem}{Theorem}
\newtheorem{corollary}{Corollary}
\newtheorem{lemma}{Lemma}
\newtheorem{prop}{Proposition}
\renewcommand{\baselinestretch}{1.2}
%\renewcommand{\thetable}{\Roman{table}}

\usepackage{hyperref}


\begin{document}

\begin{center}
\Large{
\textbf{Instructions for the supplementary codes of 'Adaptive Grids for the Estimation of Dynamic Programming Models'} \\[10pt]
}


\textbf{Authors: Andreas Lanz, Gregor Reich and Ole Wilms}\\
\textbf{Contact: \url{http://www.olewilms.com}}

\end{center}

\noindent \textbf{Instructions:}

\begin{itemize}
	\item The codes solves the bus engine replacement model of  Rust (1987) both using a fixed grid and the flexible grid proposed in the paper.
	
	\item As in Section 2.3.3 of the paper, a monte carlo study is conducted to estimate the model parameters using both, the flexible and fixed grid.
	
	\item The main file to run the codes is 'Main.m'. It reproduces Table 3 and 5 in the paper depending on the choice of cost function and $\theta_2$ parameter.
	
	\item Please use Matlab R2017a or a more recent version to run the codes. 
\end{itemize}


\end{document}
